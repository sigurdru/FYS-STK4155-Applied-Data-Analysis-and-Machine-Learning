\documentclass[reprint,english,notitlepage,aps,nobalancelastpage,nofootinbib]{revtex4-1}  % defines the basic parameters of the document
% if you want a single-column, remove reprint

% allows special characters (including æøå)
\usepackage[utf8]{inputenc}
%\usepackage [norsk]{babel} %if you write norwegian
\usepackage[english]{babel}  %if you write english
\usepackage[T1]{fontenc}


%% note that you may need to download some of these packages manually, it depends on your setup.
%% I recommend downloading TeXMaker, because it includes a large library of the most common packages.

\usepackage{physics,amssymb}  % mathematical symbols (physics imports amsmath)
\usepackage{graphicx, bm}         % include graphics such as plots
\usepackage{upgreek}
\usepackage{float}
\usepackage{xcolor}           % set colors
\usepackage[hidelinks]{hyperref}         % automagic cross-referencing (this is GODLIKE)
\usepackage{tikz}             % draw figures manually
\usepackage{listings}         % display code
\usepackage{subfigure}        % imports a lot of cool and useful figure commands
\usepackage{verbatim}
\usepackage{adjustbox}
\usepackage{mathpazo}
\usepackage{siunitx}
\usepackage{kantlipsum}
\usepackage{bigints}
\usepackage{multirow}
\usepackage{geometry}
\usepackage{algorithm2e}


% defines the color of hyperref objects
% Blending two colors:  blue!80!black  =  80% blue and 20% black
\hypersetup{ % this is just my personal choice, feel free to change things
    colorlinks,
    linkcolor={red!50!black},
    citecolor={blue!50!black},
    urlcolor={blue!80!black}}

%% Defines the style of the programming listing
%% This is actually my personal template, go ahead and change stuff if you want
\lstset{ %
	inputpath=,
	backgroundcolor=\color{white!88!black},
	basicstyle={\ttfamily\scriptsize},
	commentstyle=\color{magenta},
	language=Python,
	morekeywords={True,False},
	tabsize=4,
	stringstyle=\color{green!55!black},
	frame=single,
	keywordstyle=\color{blue},
	showstringspaces=false,
	columns=fullflexible,
	keepspaces=true}

\newcommand\numberthis{\addtocounter{equation}{1}\tag{\theequation}}
\newcommand{\ihat}{\boldsymbol{\hat{\textbf{\i}}}}
\newcommand{\jhat}{\boldsymbol{\hat{\textbf{\j}}}}
\newcommand{\khat}{\boldsymbol{\hat{\textbf{k}}}}
\newcommand{\del}[1]{\textbf{#1)}}
\newcommand{\svar}[1]{\underline{\underline{{#1}}}}
\newcommand{\vc}[1]{\mathbf{#1}}
\renewcommand{\deg}{^{\circ}}
\newcommand{\hksqrt}[2][]{\ \mathpalette\DHLhksqrt{[#1]{#2\,}}}
\def\DHLhksqrt#1#2{\setbox0=\hbox{$#1\sqrt#2$}\dimen0=\ht0
	\advance\dimen0-0.3\ht0
	\setbox2=\hbox{\vrule height\ht0 depth -\dimen0}
	{\box0\lower0.65pt\box2}}

\graphicspath{{../output/}} % search for figures in this dir



\begin{document}


\begin{titlepage}
	\begin{center}
	\textbf{Analysis of Regression and Resampling Methods}

	\vspace{0.2cm}
	Håkon Olav Torvik, Vetle Vikenes and Sigurd Sørlie Rustad

	\vspace{0.5cm}
	\includegraphics[scale=0.5]{../../pictures/UIO}
	\vspace{0.8cm}

	University of Oslo\\
	Norway\\
	\today	\\
	\end{center}
	\tableofcontents
	\clearpage
\end{titlepage}

\begin{abstract}
\end{abstract}

%\maketitle                              % creates the title


\section{Introduction}
Regression analysis is a statistical method for fitting a function to data. It is useful for building mathematical models to explain observations. There are several regression methods to achieve this, all with their strengths and weaknesses. We will in this paper study 3 different methods; ordinary least squares, Ridge and Lasso regression.

Larger datasets contain more information, giving more accurate models. However, real-world datasets usually have a fixed size, and getting more is practically impossible. For smaller datasets it is then useful to have tools mitigating the effects of little data. Resampling methods does this by running the same data through the regression, without over-fitting the model to the sample data. In addition to the regression methods, we will also study the effect of bootstrapping and cross-validating the data. 

In order to study this, we need data to analyze. Franke's function is a two-dimensional function resembling a topographical data. We will first do a polynomial fit with $x$ and $y$ dependence of this function. Because it is an analytical function, we can generate as much data as we need. We will therefore also do the same on topographical data from the real world, where the amount of data is limited. 

\section{Ordinary Least Square}


\section{Bias-variance trade-off and Bootstrapping}


\section{Cross-validation}


\section{Ridge Regression}


\section{Lasso regression}


\section{Analysis of real data}


\begin{thebibliography}{}
\bibitem[]{ref} Ref.

\end{thebibliography}


\end{document}
