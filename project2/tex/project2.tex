\documentclass[12pt]{extarticle}
\usepackage[english]{babel}
\usepackage{NotesTeX}
\usepackage{subfigure}
\usepackage{tikz}
\usetikzlibrary{arrows}
\usepackage{multirow}
\usepackage{listings}
\usepackage{extarrows}
\usepackage{parskip}
\usepackage{eurosym}

\collaborationImg{\includegraphics[width=30mm]{../../pictures/UIO.png}}

\author{\Large Håkon Olav Torvik, Vetle Vikenes and Sigurd Sørlie Rustad} 
\title{\Huge Project 2}
\affiliation{\large FYS-STK4155 – Applied Data Analysis and Machine Learning
\\Autumn 2021\\Department of Physics\\University of Oslo\\\\\today}
\begin{document}
\abstract{
	The diffusion equation is used to predict the evolution of many phenomena. In this report we are working with a constant collective diffusion coefficient, giving us a linear differential equation identical to the heat equation. We study it in both one and two dimensions, for each comparing numerical results with analytic results for various values of parameters such as the the number of integration points and time step size. In one dimension we use explicit forward Euler, implicit backward Euler and implicit Crank-Nicolson, and in two dimensions only forward Euler.
}
\maketitle

\section{Introduction}
\section{Theory}
\section{Methods}
\section{Results}
\section{Discussion}
\section{Conclusion}
\appendix
\section{Appendix}
\begin{thebibliography}{}
	\bibitem[]{lectures2015} Morten Hjorth-Jensen, Computational Physics, Lecture Notes Fall 2015, August 2015, https://github.com/CompPhysics/ComputationalPhysics/blob/master/doc/Lectures/lectures2015.pdf.
\end{thebibliography}

\end{document}

