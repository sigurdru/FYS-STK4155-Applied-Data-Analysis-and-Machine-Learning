\documentclass[12pt]{extarticle}
\usepackage[english]{babel}
\usepackage{NotesTeX}
\usepackage{subfigure}
\usepackage{tikz}
\usetikzlibrary{arrows}
\usepackage{multirow}
\usepackage{listings}
\usepackage{extarrows}
\usepackage{parskip}
\usepackage{eurosym}
\usepackage{footmisc}

\graphicspath{{../output/}}
\collaborationImg{\includegraphics[width=30mm]{../../pictures/UIO.png}}

\author{\Large Vetle Nevland, Vetle Vikenes \& Sigurd Sørlie Rustad}
\title{\Huge Machine Learning: Bra Tittel}
\affiliation{\large FYS-STK4155 – Applied Data Analysis and Machine Learning
\\Autumn 2021\\Department of Physics\\University of Oslo\\\\\today}
\begin{document}
\abstract{Coming soon!
}
\maketitle
\pagestyle{myplain}

\section{Introduction}

We will in no way answer all questions linked to the aforementioned methods. So that anyone can reproduce or continue our studies, we list all the code, results and instructions on running the code in our GitHub repository\footnote{\href{https://github.com/sigurdru/FYS-STK4155/tree/main/project2}{https://github.com/sigurdru/FYS-STK4155/tree/main/project3}}.

\section{Theory}
In the theory-section we aim to give a brief explanation of the main concepts and terminology used in this report. For a more in-depth explanation we recommend reading the appropriate sections in \cite{2019}, which has been of great inspiration and help for us throughout the project.


\subsection*{The diffusion equation}

The full diffusion equation reads
\begin{equation}
\frac{\partial u(\mathbf{r}, t)}{\partial t} = \nabla \cdot \left[D(u, \mathbf{r})\nabla u(\mathbf{r}, t)\right],
\end{equation}
where $\mathbf{r}$ is a positional vector and $D(u,r)$ the collective diffusion coefficient. If $D(u,\mathbf{r}) = 1$ the equation simplifies to a linear differential equation
\begin{equation}
\frac{\partial u}{\partial t} = \nabla^2u(\mathbf{r}, t),
\end{equation}
or
\begin{equation}
\label{eq:diffusion_equation}
\left(\frac{\partial^2}{\partial x^2} + \frac{\partial^2}{\partial y^2} + \frac{\partial^2}{\partial z^2}\right) u(x,y,z,t) = \frac{\partial u(x,y,z,t)}{\partial t}
\end{equation}
in cartesian coordinates. In this report we are going to study a one dimensional rod of length $L=1$. I.e. we need the one dimensional diffusion equation
\begin{align}
\label{eq:diffusion_equation_1D}
\frac{\partial^2 u(x,t)}{\partial x^2} &= \frac{\partial u(x,t)}{\partial t},
\end{align}
with boundary conditions
\begin{align}
u(x,0) &= \sin(\pi x) \ \ 0\leq x\leq L,\\
u(0,t) &= 0 \ \ t\geq 0 \text{ and} \\
u(L,t) &= 0 \ \ t\geq 0.
\end{align}

\subsection*{Explicit forward Euler}
In this section we want to cover the explicit forward Euler. By explicit we mean that the value at the next grid point is determined entirely by known or previously calculated values.

The one-dimensional diffusion equation \eqref{eq:diffusion_equation_1D} reads 
\begin{equation}
\frac{\partial^2u(x, t)}{\partial x^2} = \frac{\partial u(x,t)}{\partial t} \ \ \text{or} \ \ u_{xx} = u_t.
\end{equation}
In this report we are going to study a one dimensional rod of length $L=1$, with boundary conditions
\begin{align}
	u(x,0) &= \sin(\pi x) \ \ 0\leq x\leq L,\\
	u(0,t) &= 0 \ \ t\geq 0 \text{ and} \\
	u(L,t) &= 0 \ \ t\geq 0.
\end{align}

To approximate the solution, we have to discretize the position and time coordinates. We can choose  $\Delta x = L/N$ and $\Delta t$ as small steps in $x$-direction and time, where $N$ are the number of discretized points in $x$-direction. Then we can define the value domain of $t$ and $x$,
\begin{equation*}
t_j = j\Delta t, \ \ j\in \mathbb{N}_0 \ \ \wedge \ \ x_i = i\Delta x, \ \ \{i \in \mathbb{N}_0 | i \leq N\}.
\end{equation*}

The algorithm for explicit forward Euler in one dimension (from \cite{lectures2015} chapter 10.2.1) reads
\begin{equation}
\label{eq:forward_euler}
u_{i, j+1} = \alpha u_{i-1, j} + (1 - 2\alpha) u_{i,j} + \alpha u_{i+1, j}
\end{equation}
where
\begin{equation*}
\alpha = \frac{\Delta t}{\Delta x^2}.
\end{equation*}
This has a local approximate error of $O(\Delta t)$ and $O(\Delta x ^2)$. 
\section{Method}
\subsection{Forward Euler scheme}
We are to use the forward Euler method to discretize the diffusion equation to be solved numerically. The forward Euler is an explicit scheme, meaning that the derivative in time is approximated at the current time level.
That is, the derivative is discretized only by known values. As a consequence, the discretized diffusion equation can be explicitly be solved for the next time step without the need of any matrix inversion to arrive at a coupled
set of discretized equations. The drawback of explicit methods is that it is less stable than implicit schemes, which approximate the derivative at the next time step.
If the derivative is calculated at the current time level, we miss any information about how the solution changes at the next time level. 
Hence, if the gradient of the next time level is significantly different than of the current time level, the numerical solution may deviate considerably from the true solution. 
If the deviation is large enough, further iterations could potentially cause instable solutions in the sense that it diverges.

\section{Results}
\section{Discussion}
\section{Conclusion}

\bibliographystyle{plain}
\bibliography{refs}
\end{document}

