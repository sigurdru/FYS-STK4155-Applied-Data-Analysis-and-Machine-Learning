\documentclass[12pt]{extarticle}
\usepackage[english]{babel}
\usepackage{NotesTeX}
\usepackage{subfigure}
\usepackage{tikz}
\usetikzlibrary{arrows}
\usepackage{multirow}
\usepackage{listings}
\usepackage{extarrows}
\usepackage{parskip}
\usepackage{eurosym}
\usepackage{footmisc}

\graphicspath{{../output/}}
\collaborationImg{\includegraphics[width=30mm]{../../pictures/UIO.png}}

\author{\Large Vetle Nevland, Vetle Vikenes \& Sigurd Sørlie Rustad}
\title{\Huge Machine Learning: Bra Tittel}
\affiliation{\large FYS-STK4155 – Applied Data Analysis and Machine Learning
\\Autumn 2021\\Department of Physics\\University of Oslo\\\\\today}
\begin{document}
\abstract{Coming soon!
}
\maketitle
\pagestyle{myplain}

\section{Introduction}

We will in no way answer all questions linked to the aforementioned methods. So that anyone can reproduce or continue our studies, we list all the code, results and instructions on running the code in our GitHub repository\footnote{\href{https://github.com/sigurdru/FYS-STK4155/tree/main/project2}{https://github.com/sigurdru/FYS-STK4155/tree/main/project3}}.

\section{Theory}
In the theory-section we aim to give a brief explanation of the main concepts and terminology used in this report. For a more in-depth explanation we recommend reading the appropriate sections in \cite{2019}, which has been of great inspiration and help for us throughout the project.


\subsection*{The diffusion equation}

The full diffusion equation reads
\begin{equation}
\frac{\partial u(\mathbf{r}, t)}{\partial t} = \nabla \cdot \left[D(u, \mathbf{r})\nabla u(\mathbf{r}, t)\right],
\end{equation}
where $\mathbf{r}$ is a positional vector and $D(u,r)$ the collective diffusion coefficient. If $D(u,\mathbf{r}) = 1$ the equation simplifies to a linear differential equation
\begin{equation}
\frac{\partial u}{\partial t} = \nabla^2u(\mathbf{r}, t),
\end{equation}
or
\begin{equation}
\label{eq:diffusion_equation}
\left(\frac{\partial^2}{\partial x^2} + \frac{\partial^2}{\partial y^2} + \frac{\partial^2}{\partial z^2}\right) u(x,y,z,t) = \frac{\partial u(x,y,z,t)}{\partial t}
\end{equation}
in cartesian coordinates. In this report we are going to study a one dimensional rod of length $L=1$. I.e. we need the one dimensional diffusion equation
\begin{align}
\label{eq:diffusion_equation_1D}
\frac{\partial^2 u(x,t)}{\partial x^2} &= \frac{\partial u(x,t)}{\partial t},
\end{align}
with boundary conditions
\begin{align}
u(x,0) &= \sin(\pi x) \ \ 0\leq x\leq L,\\
u(0,t) &= 0 \ \ t\geq 0 \text{ and} \\
u(L,t) &= 0 \ \ t\geq 0.
\end{align}

\subsection*{Analytical solution}
An analytical solution of the 1D diffusion equation can be dervied using the method of separation of variables.
\[ u(x,t) = X(x)T(t) \]
The solution is separated into a function $X$ only depending on the independent variable $x$, and a function $T$ only depending on the independent variable $t$. Equation \ref{eq:diffusion_equation_1D} can then be rewritten as

\begin{align*}
	\frac{\partial^2 X(x)T(t)}{\partial x^2} &= \frac{\partial X(x)T(t)}{\partial t} \\
	T(t)\frac{\partial^2 X(x)}{\partial x^2} &= X(x)\frac{\partial T(t)}{\partial t} \\
	\frac{1}{X}\frac{\partial^2 X(x)}{\partial x^2} &= \frac{1}{T}\frac{\partial T(t)}{\partial t}
\end{align*}

The core of the method now becomes clear. The independent variables are separated and put on either side of the equation. Because $x$ and $t$ are independent we may fix one of them, say $x$, while letting the other ($t$) vary. The left side of the equation is thus constant, and since we have equality the expression on the right side must equal the same constant, for arbitrary $t$. Therefore, we can set the left side and right side equal to a constant $-k^2$. The reason for defining a negative constant is to prevent a growing solution, which will be clear on the derivation.

\[ \frac{1}{X}\frac{\partial^2 X(x)}{\partial x^2} = \frac{1}{T}\frac{\partial T(t)}{\partial t} = -k^2\]

This is solved for the functions $X$ and $T$ separately. ...

\subsection*{Explicit forward Euler}
In this section we want to cover the explicit forward Euler. By explicit we mean that the value at the next grid point is determined entirely by known or previously calculated values.

The one-dimensional diffusion equation \eqref{eq:diffusion_equation_1D} reads 
\begin{equation}
\frac{\partial^2u(x, t)}{\partial x^2} = \frac{\partial u(x,t)}{\partial t} \ \ \text{or} \ \ u_{xx} = u_t.
\end{equation}
In this report we are going to study a one dimensional rod of length $L=1$, with boundary conditions
\begin{align}
	u(x,0) &= \sin(\pi x) \ \ 0\leq x\leq L,\\
	u(0,t) &= 0 \ \ t\geq 0 \text{ and} \\
	u(L,t) &= 0 \ \ t\geq 0.
\end{align}

To approximate the solution, we have to discretize the position and time coordinates. We can choose  $\Delta x = L/N$ and $\Delta t$ as small steps in $x$-direction and time, where $N$ are the number of discretized points in $x$-direction. Then we can define the value domain of $t$ and $x$,
\begin{equation*}
t_j = j\Delta t, \ \ j\in \mathbb{N}_0 \ \ \wedge \ \ x_i = i\Delta x, \ \ \{i \in \mathbb{N}_0 | i \leq N\}.
\end{equation*}

The algorithm for explicit forward Euler in one dimension (from \cite{lectures2015} chapter 10.2.1) reads
\begin{equation}
\label{eq:forward_euler}
u_{i, j+1} = \alpha u_{i-1, j} + (1 - 2\alpha) u_{i,j} + \alpha u_{i+1, j}
\end{equation}
where
\begin{equation*}
\alpha = \frac{\Delta t}{\Delta x^2}.
\end{equation*}
This has a local approximate error of $O(\Delta t)$ and $O(\Delta x ^2)$. \\

We are to use the forward Euler method to discretize the diffusion equation to be solved numerically. The forward Euler is an explicit scheme, meaning that the derivative in time is approximated at the current time level.
That is, the derivative is discretized only by known values. As a consequence, the discretized diffusion equation can be explicitly be solved for the next time step without the need of any matrix inversion to arrive at a coupled
set of discretized equations. The drawback of explicit methods is that it is less stable than implicit schemes, which approximate the derivative at the next time step.
If the derivative is calculated at the current time level, we miss any information about how the solution changes at the next time level. 
Hence, if the gradient of the next time level is significantly different than of the current time level, the numerical solution may deviate considerably from the true solution. 
If the deviation is large enough, further iterations could potentially cause instable solutions in the sense that it diverges.
\par The forward Euler scheme in general requires a quite low time step $\Delta t$ to produce stable solutions. Small time steps are convenient at the start of the diffusion process when the solutions changes quickly. However, when approaching the stationary limit the solution changes very slow and so does the update when using small $\Delta t$. This limitation is resolved by implicit methods \cite{Linge2017}.

\section{Method}
\subsection*{Unit testing}
Before the numerical explicit scheme is applied to a particular problem, it is important to test that the discretized equations return expected results. Unit tests are constructed to test the implementation . This is done by manually calculating the solution of the first two time steps given the initial condition, and comparing the result with that obtained by the numerical scheme. Since the same recursive formula is used for all time steps, it is sufficient to test the two first time steps. To avoid too much computations we choose five equally sized intervals between $x=0$ and $x=L=1$, that is $\Delta x = 0.2$. The time step is set to $\Delta t = 0.01$, ensuring stability.
For the first time step $j=1$, that is $t=\Delta t$, we have the following two boundary values and four interior points

\begin{align*}
	u_0^1 &= u_5^1 = 0 \\
	u_i^1 &= \frac{\Delta t}{\Delta x^2}u_{i+1}^0 + (1 - 2\frac{\Delta t}{\Delta x^2})u_i^0 + \frac{\Delta t}{\Delta x^2}u_{i-1}^0 \\
	&= 0.05u_{i+1}^0 + 0.9u_i^0 + 0.05u_{i-1}^0
\end{align*}

Using the initial condition $u_*^0 = \big(\sin(0),\:\sin(0.2\pi),\:\sin(0.4\pi),\:\sin(0.6\pi), \:\sin(0.8\pi),\:\sin(\pi)\big)$, we get

\begin{align*}
	u_0^1 &= 0 \\
	u_1^1 &= 0.531656755 \\
	u_2^1 &= 0.8602387 \\
	u_3^1 &= 0.8602387 \\
	u_4^1 &= 0.531656755 \\
	u_5^1 &= 0
\end{align*}

The forward euler scheme prints out the following values for the first time step
\[ (0.,\:         0.53165676,\: 0.8602387,\:  0.8602387,\:  0.53165676,\: 0.        ) \]
Hence, the result from forward euler is equal to the manual calulations up to machine precision.

\par Using the values from the first time step, we can calculate the solution at second time step. The recursive formula makes the manual calculations straightforward.

\begin{align*}
	u_0^2 &= 0 \\
	u_1^2 &= 0.480888053 \\
	u_2^2 &= 0.778093215 \\
	u_3^2 &= 0.778093215 \\
	u_4^2 &= 0.480888053 \\
	u_5^2 &= 0
\end{align*}

The forward euler scheme gives the following result for the second time step
\[ (0.,\:         0.48088805,\: 0.77809321,\: 0.77809321,\: 0.48088805,\: 0.        ) \]
Once again, the two approaches provide the same result up to machine precision. Thus, the internal functionality of forward euler is validated. It remains to verify the accuracy of the scheme by comparing it with the analytical solution of the diffusion equation \ref{eq:diffusion_equation_1D} with provided initial and boundary conditions.

\subsection*{Accuracy assessment}
The unit test shows that the forward euler scheme works as expected, but to what accuracy can it reproduce the analytical solution? To address this we compute the deviation of the numerical solution to the analytical solution for two different mesh resolutions in space, $\Delta x=0.1$ and $\Delta x = 0.01$. For comparison purposes the time step is adjusted in each case to ensure stability, and is defined as $\Delta t = 0.3\cdot \Delta x^2$. The error is computed and accumulated for a fixed number of time steps $N$ in both cases. The applied error measure at a given time step is simply the maximum absolute difference between the numerical solution $u$ and the analytical solution $u_e$ across the entire spatial domain.
\[ e^n = \argmax_{i \in [0,N_x]} |u_i^n - u_{e,i}^n| \]

The total accumulated error is then
\[ E = \sum_{n=1}^N e^n, \]
which is the error measure to be used when testing how well the numerical solution approximates the analytical solution.



\section{Results}

\subsection*{Error of Forward Euler scheme}

To assess the accuracy of our implemented numerical scheme, we need to test the how well the produced solution fits the analytical solution. In this project we use the accumulated maximum absolute error, as described in Methods.

\begin{table}[h]
	\centering
	\begin{tabular}{|l|l|l|l|}
		\hline
		\textbf{n}      & \textbf{5} & \textbf{10} & \textbf{50} \\ \hline
		$\Delta x=0.1$  & 0.0061     & 0.0155       & 0.0838       \\ \hline
		$\Delta x=0.01$ & 6.073$\cdot 10^{-5}$      & 1.536$\cdot 10^{-4}$       &   8.302$\cdot 10^{-4}$     \\ \hline
	\end{tabular}
\caption{Accumulated maximum absolute error between numerical solution and analytical solution of the 1D diffusion equation. The test is performed for two different spatial resolutions and for three number of time steps.}
\label{tab:errors}
\end{table}

The test of accumulated error for $\Delta x=0.1$ and $\Delta x=0.01$ for different number of time steps $n$ is shown in Table \ref{tab:errors}. The accumulated error obviously increases when evaluating the error for more time steps because more individual errors are accumulated to the total error. Another feature is that reducing the spatial step $\Delta x$ significantly reduces the accumulated error. For instance, from Table \ref{tab:errors} we have that $0.838/8.302\cdot 10^{-4}=100.94$. Approximately the same factor is reproduced by the other time steps $n$ as well. This shows that reducing $\Delta x$ by a factor of 10 will reduce the accumulated error by a factor of 100. Hence, refining the spatial resolution will significantly improve the performance of the forward euler scheme. The results strongly indicate that the numerical solution converges to the analytical solution as $\Delta x \rightarrow 0$. 

\begin{figure}[h]
	\includegraphics[scale=0.4]{}
\end{figure}

\par A plot of the accumulated error for $\Delta x=0.01$ is shown in Figure \ref{fig:error_dx001_n50}. Notice that the error for time step $n=0$ is zero. This is because the numerical solution for the first time step is simply the initial condition, yielding a perfect correspondance between the numerical and analytical solution. As time proceeds, the error quickly increases and reaches a maximum at around time $t=0.1$. At this timestep, the solution includes significant gradients. The Forward euler discretization of the derivative is susceptible to large gradients, so one expects the numerical scheme to contain larger errors, as evidenced in Figure \ref{fig:error_dx001_n50}. At later stages though, the solution becomes more linear and the gradients mitigate. Thus, forward euler is capable of reproducing the analytical solution to a much better approximation. As time goes to infinity the error converges to zero.


\section{Discussion}
\section{Conclusion}

\bibliographystyle{plain}
\bibliography{refs}
\end{document}

